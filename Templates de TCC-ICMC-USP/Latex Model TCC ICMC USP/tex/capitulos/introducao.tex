\section{Motivação e Contextualização}

\section{Objetivos}

\section{Organização}



% Comando simples para exibir comandos Latex no texto
\newcommand{\comando}[1]{\textbf{$\backslash$#1}}

Este documento explica brevemente como trabalhar com a classe \LaTeX~\textit{icmc} para confeccionar trabalhos acadêmicos seguindo as normas da \sigla{ABNT}{Associação Brasileira de Normas Técnicas} e as \aspas{\textit{Diretrizes para apresentação de dissertações e teses da USP: documento eletrônico e impresso. Parte I (ABNT)}}, publicado pelo \sigla{SIBi}{Sistema Integrado de Bibliotecas} USP. O presente manual também atende as exigências prevista no regimento do Programa de Pós-graduação em \sigla{CCMC}{Ciências da Computação e Matemática Computacional} do \sigla{ICMC}{Instituto de Ciências Matemáticas e de Computação} da \sigla{USP}{Universidade de São Paulo}.


A classe \textit{icmc} foi construída com base na última versão da classe \textit{abntex2} e do pacote \textit{abntex2cite}. Portanto, este documento exemplifica a elaboração de trabalho
acadêmico (tese, dissertação e outros do gênero) produzido conforme a ABNT NBR
14724:2011 \textit{Informação e documentação - Trabalhos acadêmicos - Apresentação}.

Assim, é altamente recomendável que seja consultada a documentação do \textit{abntex2}\footnote{https://code.google.com/p/abntex2/downloads/list}. A classe \textit{abntex2} foi desenvolvida para facilitar a escrita de documentos seguindo as normas da ABNT no ambiente \LaTeX\;\cite{frasson:2005:classe_abnt}.

Todo o trabalho de pesquisa e ajustes da presente classe \LaTeX~\emph{icmc} foram feitos pelo aluno mestrado do Programa de Pós-graduação em Ciência da Computação e Matemática Computacional, Humberto Lidio Antonelli, durante a confecção da sua monografia de qualificação.

O requisito básico para utilização da classe \textit{icmc} é criar um documento desta classe com o comando
\comando{documentclass[@parameters]\{icmc\}} e ter, no diretório de trabalho, o arquivo \emph{icmc.cls} presente. Entretanto, recomenda-se fortemente manter a estrutura de diretório inicial fornecida por este modelo.

Os parâmetros possíveis utilizados pelo \comando{documentclass} são:
\begin{description}
\item[french, spanish, english, brazil] Adiciona o idioma para correta hifenização correta no documento. O último idioma declarado é o principal do documento. O valor padrão é \textbf{brazil}.
\end{description}

