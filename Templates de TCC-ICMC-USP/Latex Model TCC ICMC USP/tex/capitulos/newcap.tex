\section{Primeira seção}

\subsection{Primeira subseção}

Na figura \ref{fig:teste} ..

\begin{figure}[!htb]
		\caption{Logomarca da USP}
		\label{fig:teste}
		\centering
		\includegraphics[width=0.3\columnwidth]{images/usp-logo}
		\caption*{Fonte: .....}
\end{figure}

Na tabela \ref{tab:teste} ..

\begin{table}[!htb]
\centering
\fontsize{8.4pt}{8.4pt}\selectfont
\def\arraystretch{2}
\addtolength{\tabcolsep}{-1pt} 
\caption{Tabela de teste}
\begin{tabular}{@{}lccccccc@{}}
\toprule
\textbf{Métodos} & \textbf{Metrica 1} & \textbf{Metrica 2} & \textbf{Metrica 3} & \textbf{Metrica 1} & \textbf{Metrica 2} & \textbf{Metrica 3} & \textbf{Metrica 1} \\ \midrule
\textbf{a}       & 0.2                & 0.4                & 0.5                & 0.2                & 0.4                & 0.5                & 0.2                \\
\textbf{s}       & 0.2                & 0.4                & 0.5                & 0.2                & 0.4                & 0.5                & 0.2                \\
\textbf{b}       & 0.2                & 0.4                & 0.5                & 0.2                & 0.4                & 0.5                & 0.2                \\ \bottomrule
\end{tabular}
\label{tab:teste}
\end{table}

Na equação \ref{eq:teste} ... 

\begin{equation} \label{eq:teste}
    E(x) = \sum_{i = 1}^{N}, 
\end{equation}
\noindent em que x ==, N === ... 

Afirmação de teste \cite{perera2019learning} .... Segundo \citeonline{perera2019learning}